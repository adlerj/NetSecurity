\chapter{Proposal: Network Threat Analysis and Deterrence Automation}

\section{Motivation}

One of the major issues confronting all computer networks today is cyber security; the goal of our
Network Threat Analysis and Deterrence Automation capstone is to build an adaptive security system
for large-scale switch networks to adapt and thwart attempts of exploitation and intrusion by
analyzing TCP/IP packets and deploying a flexible response (such as the restriction of traffic via
iptables or outright blocking the origin of offending packets) based on perceived threats over network.
Penetration methods and exploitable vulnerabilities are evolving at an alarming rate.
Increasingly dangerous practices such as the ``Advanced Persistent Threat'' (APT), a long-term
network attack utilizing a plethora of effective methods to break into a system over a prolonged period
of time, are becoming an increasingly intimidating issue confronting enterprise networks. Our threat
analysis system will be a proof-of-concept for a stronger, more cooperative network security
environment.
Our intention is to start small and build a simple computing operating system that may be
placed on a management role over a switch network that can preventively detect and deter brute
force dictionary attacks, one of the easiest but most expensive attacks that can be utilized against a
network. From there, we will expand our system to actively analyze, detect, and deter more
sophisticated attacks (such as distributed denial of service attacks) by employing distributed
computing practices to parallelize TCP/IP threat analysis and corresponding system policy changes to
deter such attacks before they successfully penetrate a network.
\emph{
There is a popular misconception that network attacks can only be deterred after
successful intrusion; we would like to challenge that misconception with our Network
Threat Analysis and Deterrence Automation capstone.
}

\subsection{Sources of Domain Knowledge}

\begin{itemize}
	\item \emph{Val A. Red} is a system administrator for Engineering Computing Services 
	and has experience with employing automation and network policy/iptables to administrate networks.
	\item \emph{Eric Cuiffo}, \emph{Parth Desai}, and \emph{Val A. Red} have programmed
	 a multithreaded SSH attack and successfully found a counter to it that can be easily implemented in any system.
	\item Professor Parashar coauthored 
	``\href{http://nsfcac.rutgers.edu/TASSL/Papers/ddos-icics-04.pdf}{Cooperative Mechanism Against DDoS Attacks},''
	 which addresses a very specific type of attack and applies a process similar to what we would like to achieve 
	 on a large scale for a large number of known attack methods.
\end{itemize}

\section{Abstract}

With the advent of increasingly potent methods employed against cyber security such as
GPU-based penetration attempts and the Advanced Persistent Threat (APT), there is an increasing
need for network administrators to be cognizant of deploying, automating, and maintaining robust
systems for managing their switches and network traffic. Due to the increasing volume and value of
data being transferred over online networks, we intend to design and develop a system to reliably
secure large networks from the switch-level with a deployable Linux kernel existing solely to monitor
and manage a switch network, applying strict, sophisticated tools and programs (iptables, etc.) to
adaptively and effectively prevent, deter, and thwart network attacks and penetration attempts

\subsection{Deliverables}

\begin{enumerate}
	\item Documentation on the following:
	\begin{enumerate}
		\item  Set-up of the customized Linux kernel, describing how to replicate our system. (Including
		system requirements, sources and dependencies, etc.)
		\item Supplementary, custom packaged programs to enhance and automate traffic sniffing and
		group policy/access over ports/protocols depending on volume and potential threat of incoming
		TCP/IP packets.
	\end{enumerate}
	
	\item A {\textbf{live}} web-facing server hosted by Engineering Computing Services (ECS) utilizing our custom
	defense system with common open ports (80 for HTTP, 22 for SSH, etc.) for testing and
	demonstration of the robustness of our system.

\end{enumerate}

\subsection{Goals}

\begin{enumerate}
\item Prepare customized Linux kernel with customized, optimized iptables flexibility for
network defense.
\item Utilize parallelized TCP/IP analysis to detect common network attacks.
\item Automate network access policy/permissions based on threat analysis.
\item Deter attacks before they succeed utilizing parallelized analysis and automated system responses.

\end{enumerate}

\subsection{Logistics}

The automation aspects of our proposed capstone may be of great interest to Juniper Networks, who
suggested a capstone very similar to what we propose with regards to the flexible changing of network
policies based on different scenarios.

\section{Anticipated Technical Challenges and Risks}

\begin{enumerate}
\item Automating network traffic sniffing (interpreting TCP/IP packets) would be expensive and one of
the most difficult aspects of our project to apply and has the risk of false positives. We will need to be
very careful writing and employing programs that operate at this level.
\item If our proposed system is somehow compromised, it would essentially put an entire network at
risk. We’ll need to ensure our system is robust and not readily accessible (employ a second access
layer, such as restricting SSH access to the system to the internal network, etc.) to reinforce the
security of the actual system.
\item There are countless kinds of attacks and every day there are more and more vulnerabilities and
software updates to compete against such vulnerabilities, so we need to very early define which
attacks we’ll address with our system first that would be reasonable for the duration of semester and
how we will approach sustainability over the long term.

\end{enumerate}

\section{Web Page}

Our web page is currently hosted via Drupal at \url{http://scarletshield.rutgers.edu}.

\section{Final Remarks}
Overall, our proposed capstone would serve as a proof-of-concept for utilizing customized,
robust operating systems to enhance switch and network management. Ideally, the combination of
our choice and employment of various tools and self-authored programs in addition to the customized
kernel should definitely contribute some insight for how network admins can flexibly apply their
technical proficiency to keep their networks safe.